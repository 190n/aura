\documentclass{article}
% GENERAL
\usepackage{setspace,graphicx,fancyhdr,hyperref,amsmath,tikz,epigraph}
\usepackage[utf8]{inputenc}

% FONT
\usepackage{arev}
\usepackage[T1]{fontenc}

% MARGINS
\usepackage[left=1in,top=1in,right=1in,bottom=1in]{geometry}
\onehalfspacing

% CUSTOM FOOTERS
\pagestyle{fancy}
\fancyhead{}                        % clear all header fields
\renewcommand{\headrulewidth}{0pt}  % no line in header area
\fancyfoot{}                        % clear all footer fields
\fancyfoot[LE,RO]{\thepage}         % for page #s
%\fancyfoot[RE,LO]{\includegraphics[scale=0.03]{adenda-logo}}

% Setting the depth for Table of Contents
\setcounter{tocdepth}{3}

\begin{document}

% --- TABLE OF CONTENTS ---
\tableofcontents
\clearpage
% -------------------------

\section{Aura 2 Design}\label{aura-2-design}

\subsection{Contents}\label{contents}

\begin{itemize}
\itemsep1pt\parskip0pt\parsep0pt
\item
  Preface
\item
  \href{/DESIGN.md\#mission-statement}{Mission Statement}
\item
  Requirements
\item
  \href{/DESIGN.md\#general-functionality}{General Functionality}

  \begin{itemize}
  \itemsep1pt\parskip0pt\parsep0pt
  \item
    \href{/DESIGN.md\#program-flow}{Program Flow}
  \item
    \href{/DESIGN.md\#dependency-resolution}{Dependency Resolution}
  \item
    \href{/DESIGN.md\#dependency-information-output}{Dependency
    Information Output}
  \item
    \href{/DESIGN.md\#concurrent-package-building}{Concurrent Package
    Building}
  \item
    \href{/DESIGN.md\#abnormal-termination}{Abnormal Termination}
  \item
    \href{/DESIGN.md\#pkginfo}{PkgInfo}
  \item
    \href{/DESIGN.md\#colour-output}{Colour Output}
  \end{itemize}
\item
  \href{/DESIGN.md\#plugins}{Plugins}

  \begin{itemize}
  \itemsep1pt\parskip0pt\parsep0pt
  \item
    \href{/DESIGN.md\#auraconf}{Aura Conf}
  \item
    \href{/DESIGN.md\#hook-list}{Hook List}
  \end{itemize}
\item
  \href{/DESIGN.md\#aesthetics}{Aesthetics}

  \begin{itemize}
  \itemsep1pt\parskip0pt\parsep0pt
  \item
    \href{/DESIGN.md\#version-information-when-upgrading}{Version
    Information when Upgrading}
  \item
    \href{/DESIGN.md\#aura-versioning}{Aura Versioning}
  \end{itemize}
\item
  \href{/DESIGN.md\#haskell-requirements}{Haskell Requirements}

  \begin{itemize}
  \itemsep1pt\parskip0pt\parsep0pt
  \item
    \href{/DESIGN.md\#strings}{Strings}
  \item
    \href{/DESIGN.md\#json-data}{JSON Data}
  \item
    \href{/DESIGN.md\#other-libraries}{Other Libraries}
  \end{itemize}
\item
  \href{/DESIGN.md\#package-requirements}{Package Requirements}
\item
  \href{/DESIGN.md\#arch-linux-specifics}{Arch Linux Specifics}
\item
  \href{/DESIGN.md\#abs-package-buildinginstallation}{ABS Package
  Building/Installation}
\item
  \href{/DESIGN.md\#aur-package-buildinginstallation}{AUR Package
  Building/Installation}
\item
  \href{/DESIGN.md\#pkgbuildadditional-build-file-editing}{PKGBUILD/Additional
  Build-file Editing}
\item
  \href{/DESIGN.md\#aur-interaction}{AUR Interaction}
\item
  \href{/DESIGN.md\#coding-standards}{Coding Standards}
\item
  \href{/DESIGN.md\#record-syntax}{Record Syntax}
\end{itemize}

\subsection{Preface}\label{preface}

This is a design document for version 2 of
\href{https://github.com/fosskers/aura}{Aura}. Note that specifications
are written in present tense, as in, ``Aura does this'' even if at the
time of writing those features aren't implemented yet. This is to ensure
that the document can act as a reference for Aura's behaviour
post-release.

\subsection{Mission Statement}\label{mission-statement}

Aura is a cross-distribution package manager for GNU/Linux systems. It
is based around a distribution-specific Hook system for custom
build/install behaviour, while maintaining a custom interface across all
distros. Aura itself provides:

\begin{itemize}
\itemsep1pt\parskip0pt\parsep0pt
\item
  Dependency management.
\item
  Package downloading.
\item
  Package-state backups/restoration.
\end{itemize}

Aura's authors recognize that \href{http://www.xkcd.com/927/}{attemping
to create universal standards can be problematic}, but that is precisely
why Aura exists. By having a unified interface over multiple packaging
standards, users can transition between distributions more easily, and
distribution developers can avoid reinventing the wheel by writing their
own package management software.

\subsection{Requirements}\label{requirements}

\subsubsection{General Functionality}\label{general-functionality}

\paragraph{Program Flow}\label{program-flow}

\textbf{This section needs reorganising}

Execution in Aura takes the following order:

\begin{enumerate}
\def\labelenumi{\arabic{enumi}.}
\itemsep1pt\parskip0pt\parsep0pt
\item
  Parse command-line options.
\item
  Collect local \texttt{Setting}s.
\item
  Branch according to capital letter operator (\texttt{-\{S,A,Q,...\}}):
\end{enumerate}

\begin{itemize}
\itemsep1pt\parskip0pt\parsep0pt
\item
  \texttt{-S \textless{}packages\textgreater{}}:

  \begin{itemize}
  \itemsep1pt\parskip0pt\parsep0pt
  \item
    A \textbf{Hook} provides functions:
  \item
    \texttt{Monad m =\textgreater{} {[}Text{]} -\textgreater{} m ({[}Text{]},{[}Package{]})}
  \item
    \texttt{Monad m =\textgreater{} Text -\textgreater{} m (Either Text Package)}
  \end{itemize}

  The former can be defined in the terms of the latter, but doesn't have
  to be if that method executes faster. The first function is given the
  names of all packages to be installed. The \texttt{{[}Text{]}} are
  packages that don't exist. They are reported.

  \begin{itemize}
  \itemsep1pt\parskip0pt\parsep0pt
  \item
    With the output of the last function, resolve dependencies by Aura's
    internal algorithm to receive:
    \texttt{Either PkgGraph {[}{[}Package{]}{]}}.
  \item
    On \texttt{Left}, analyse the given \texttt{PkgGraph}, yield output
    as described in \href{/DESIGN.md\#dependency-resolution}{Dependency
    Resolution}, and quit.
  \item
    On \texttt{Right} display a chart as described
    \href{/DESIGN.md\#version-information-when-upgrading}{here}.
  \item
    Download each package via Aura's internal algorithm.
  \item
    A \textbf{Hook} provides an install function
    \texttt{MonadError m =\textgreater{} {[}{[}Package{]}{]} -\textgreater{}   m ()}
  \end{itemize}
\item
  \texttt{-\{S,A,Q\}i \textless{}packages\textgreater{}}:

  \begin{itemize}
  \itemsep1pt\parskip0pt\parsep0pt
  \item
    Call a \textbf{Hook} that provides
    \texttt{Monad m =\textgreater{} Text -\textgreater{} m PkgInfo}. The
    contents of the \texttt{PkgInfo} ADT are described
    \href{/DESIGN.md\#pkginfo}{here}.
  \item
    Aura gives output according to the \texttt{PkgInfo}.
  \end{itemize}
\item
  \texttt{-\{S,A,Q\}s \textless{}pattern\textgreater{}}:

  \begin{itemize}
  \itemsep1pt\parskip0pt\parsep0pt
  \item
    Call a \textbf{Hook} that provides
    \texttt{Monad m =\textgreater{} Text -\textgreater{} m {[}PkgInfo{]}}.
    Where the \texttt{Text} is a pattern to be searched for.
  \item
    Aura gives output according to the \texttt{{[}PkgInfo{]}}.
  \end{itemize}
\end{itemize}

\paragraph{Dependency Resolution}\label{dependency-resolution}

\begin{itemize}
\itemsep1pt\parskip0pt\parsep0pt
\item
  AUR dependencies are no longer resolved through PKGBUILD bash parsing.
  The AUR 3.0 API includes the necessary dependency information.
\item
  \textbf{Resolution Successful}: Data in the form
  \texttt{{[}{[}Package{]}{]}} is yielded. These are groups of packages
  that may be built and installed simultaneously. That is, they are not
  interdependent in any way.
\item
  \textbf{Version Conflicts}:
\item
  Dependency resolution fails and the build does not continue.
\item
  The user is shown the chart below so it is clear what dependencies
  from what packages are causing issues.
\item
  All packages that had dependency issues are shown.
\item
  Supplying the \texttt{-\/-json} flag will output this data as JSON for
  capture by other programs.
\end{itemize}

\begin{verbatim}
| Dep Name | Parent | Status   | Version |
| -------- | ------ | -------- | ------- |
| foo      | None   | Local    | 1.2.3   |
| foo      | bar    | Incoming | < 1.2.3 |
| foo      | baz    | Incoming | > 1.2.3 |
| -------- | ------ | -------- | ------- |
| curl     | git    | Local    | 7.36.0  |
| curl     | pacman | Incoming | 7.37.0  |
| -------- | ------ | -------- | ------- |
| lua      | vlc    | Incoming | 5.2.3   |
| lua      | conky  | Incoming | 5.2.2   |
\end{verbatim}

\begin{Shaded}
\begin{Highlighting}[]
\CommentTok{// As JSON:}
\NormalTok{\{ [ \{ }\StringTok{"Name"}\NormalTok{: }\StringTok{"foo"}
    \NormalTok{, }\StringTok{"Local"}\NormalTok{: \{ }\StringTok{"Parent"}\NormalTok{: }\StringTok{"None"}
               \NormalTok{, }\StringTok{"Version"}\NormalTok{: }\StringTok{"1.2.3"} \NormalTok{\}}
    \NormalTok{, }\StringTok{"Incoming"}\NormalTok{: [ \{ }\StringTok{"Parent"}\NormalTok{: }\StringTok{"bar"}
                    \NormalTok{, }\StringTok{"Version"}\NormalTok{: }\StringTok{"< 1.2.3"} \NormalTok{\}}
                  \NormalTok{, \{ }\StringTok{"Parent"}\NormalTok{: }\StringTok{"baz"}
                    \NormalTok{, }\StringTok{"Version"}\NormalTok{: }\StringTok{"> 1.2.3"} \NormalTok{\}}
                  \NormalTok{]}
    \NormalTok{\}}
  \NormalTok{, \{ }\StringTok{"Name"}\NormalTok{: }\StringTok{"curl"}
    \NormalTok{, }\StringTok{"Local"}\NormalTok{: \{ }\StringTok{"Parent"}\NormalTok{: }\StringTok{"git"}
               \NormalTok{, }\StringTok{"Version"}\NormalTok{: }\StringTok{"7.36.0"} \NormalTok{\}}
    \NormalTok{, }\StringTok{"Incoming"}\NormalTok{: [ \{ }\StringTok{"Parent"}\NormalTok{: }\StringTok{"pacman"}
                    \NormalTok{, }\StringTok{"Version"}\NormalTok{: }\StringTok{"7.37.0"} \NormalTok{\}}
                  \NormalTok{]}
    \NormalTok{\}}
  \NormalTok{, \{ }\StringTok{"Name"}\NormalTok{: }\StringTok{"lua"}
    \NormalTok{, }\StringTok{"Local"}\NormalTok{: }\StringTok{"None"}
    \NormalTok{, }\StringTok{"Incoming"}\NormalTok{: [ \{ }\StringTok{"Parent"}\NormalTok{: }\StringTok{"vlc"}
                    \NormalTok{, }\StringTok{"Version"}\NormalTok{: }\StringTok{"5.2.3"} \NormalTok{\}}
                  \NormalTok{, \{ }\StringTok{"Parent"}\NormalTok{: }\StringTok{"conky"}
                    \NormalTok{, }\StringTok{"Version"}\NormalTok{: }\StringTok{"5.2.2"} \NormalTok{\}}
                  \NormalTok{]}
    \NormalTok{\}}
  \NormalTok{]}
\NormalTok{\}}
\end{Highlighting}
\end{Shaded}

\paragraph{Dependency Information
Output}\label{dependency-information-output}

\begin{itemize}
\itemsep1pt\parskip0pt\parsep0pt
\item
  Information for all immediate dependencies for any given package can
  be output in human-readable format by default with \texttt{-\{A,S\}d}.
\item
  Adding \texttt{-\/-recursive} will yield all dependencies and
  \emph{their} dependencies as well.
\item
  Adding \texttt{-\/-json} will output this information in JSON for use
  by other software that may sit on top of Aura.
\end{itemize}

\paragraph{Concurrent Package
Building}\label{concurrent-package-building}

\begin{itemize}
\itemsep1pt\parskip0pt\parsep0pt
\item
  Package data is returned from dependency checking in the form
  \texttt{{[}{[}Package{]}{]}} (see
  \href{/DESIGN.md\#dependency-resolution}{Dependency Resolution}). Each
  sublist of packages have no interdependencies, so they are built
  concurrent to each other and then installed as a block.
\end{itemize}

\paragraph{Package State Backups}\label{package-state-backups}

\begin{itemize}
\itemsep1pt\parskip0pt\parsep0pt
\item
  \texttt{aura -B} stores a snapshot of all currently installed packages
  and their versions in \texttt{/var/cache/aura/states}.
\item
  Filenames are of the form: \texttt{YYYY.MM(MonthName).DD.HH.MM}
\item
  The data itself is stored as JSON to ease use by other tools:
\end{itemize}

\begin{Shaded}
\begin{Highlighting}[]
\NormalTok{\{ }\StringTok{"date"}\NormalTok{: }\StringTok{"2014-04-09"}
\NormalTok{, }\StringTok{"time"}\NormalTok{: }\StringTok{"20:00"}
\NormalTok{, }\StringTok{"packages"}\NormalTok{: [ \{ }\StringTok{"pkgname"}\NormalTok{: }\StringTok{"alsa-lib"}
                \NormalTok{, }\StringTok{"version"}\NormalTok{: }\StringTok{"1.0.27.2-1"} \NormalTok{\}}
                \CommentTok{// more packages here}
              \NormalTok{]}
\NormalTok{\}}
\end{Highlighting}
\end{Shaded}

\paragraph{PkgInfo}\label{pkginfo}

\begin{itemize}
\itemsep1pt\parskip0pt\parsep0pt
\item
  \texttt{-\{S,A,Q\}i} yields \texttt{PkgInfo} data. It holds:
\item
  Repository name
\item
  Package name
\item
  Version
\item
  Description
\item
  Architecture
\item
  URL
\item
  Licenses
\item
  ``Provides''
\item
  Dependencies
\item
  ``Conflicts With''
\item
  Maintainer
\item
  Optional fields (provided as \texttt{{[}(Text,Text){]}}):

  \begin{itemize}
  \itemsep1pt\parskip0pt\parsep0pt
  \item
    Download/Install sizes
  \item
    Group
  \item
    Votes
  \item
    GPG information
  \item
    etc.
  \end{itemize}
\end{itemize}

\paragraph{Abnormal Termination}\label{abnormal-termination}

\begin{itemize}
\itemsep1pt\parskip0pt\parsep0pt
\item
  Users can halt Aura with \texttt{Ctrl-d}. The message
  \texttt{Stopping Aura...} is shown. All temporary files in use are
  cleared here.
\end{itemize}

\paragraph{Colour Output}\label{colour-output}

\begin{itemize}
\itemsep1pt\parskip0pt\parsep0pt
\item
  All output to terminal (save JSON data) is output in colour where
  appropriate. The user can disable this with
  \texttt{-\/-no-colo\{ur,r\}}
\end{itemize}

\subsubsection{Plugins}\label{plugins}

Like XMonad, behaviour is built around hooks/plugins that are themselves
written in Haskell. Each Linux distribution writes and provides to
\texttt{AuraConf.hs} functions that fill certain type/behaviour
requirements as explained below.

\paragraph{AuraConf}\label{auraconf}

\texttt{AuraConf.hs} is Aura's configuration file. It is typically
located in \texttt{TODO: LOCATION HERE}. Here, distributions and users
can add Hooks to define custom behaviour for their native packaging
system. The command \texttt{aura -\/-recompile} rebuilds Aura with new
Hooks. Also, the following paths can be defined in this file:

\begin{itemize}
\itemsep1pt\parskip0pt\parsep0pt
\item
  Package cache.
\item
  Aura log file.
\item
  Default build directory.
\item
  Mirror URLs for binary downloads.
\item
  TODO: What else?
\end{itemize}

\paragraph{Hook List}\label{hook-list}

Pending.

\subsubsection{Aesthetics}\label{aesthetics}

\paragraph{Version Information when
Upgrading}\label{version-information-when-upgrading}

\begin{itemize}
\itemsep1pt\parskip0pt\parsep0pt
\item
  Need a nice chart.
\end{itemize}

\paragraph{Aura Versioning}\label{aura-versioning}

\begin{itemize}
\itemsep1pt\parskip0pt\parsep0pt
\item
  Aura uses \href{http://semver.org/}{Semantic Versioning}, meaning it's
  version numbers are of the form \texttt{MAJOR.MINOR.PATCH}.
\end{itemize}

\subsubsection{Haskell Requirements}\label{haskell-requirements}

\paragraph{Strings}\label{strings}

\begin{itemize}
\itemsep1pt\parskip0pt\parsep0pt
\item
  All Strings are represented as \texttt{Text} from \texttt{Data.Text}.
  This is available in the \texttt{text} package from Hackage.
\end{itemize}

\begin{Shaded}
\begin{Highlighting}[]
\OtherTok{\{-# LANGUAGE OverloadedStrings #-\}}
\end{Highlighting}
\end{Shaded}

should be used where appropriate for String literals being converted to
Text automatically.

\paragraph{JSON Data}\label{json-data}

\begin{itemize}
\itemsep1pt\parskip0pt\parsep0pt
\item
  All JSON input and output is handled through \texttt{aeson} and
  \texttt{aeson-pretty}.
\end{itemize}

\paragraph{Other Libraries}\label{other-libraries}

Information on other Hackage libraries used in Aura can be found
\href{https://github.com/fosskers/aura/issues/223}{here}.

\subsubsection{Package Requirements}\label{package-requirements}

Aura must be available in the following forms: - \texttt{haskell-aura}
An AUR package pulled from Hackage, with all special install
instructions contained in \texttt{Setup.hs}. - \texttt{aura} What was
\texttt{aura-bin} in Aura 1. A pre-built binary for those with no
interest in Haskell. The old \texttt{aura-bin} package will be noted as
depreciated, left as Aura 1, and removed from the AUR \textbf{two}
months after the release of Aura 2. - \texttt{aura-git} the same as is
currently available. Should man page install instructions, etc., be in
\texttt{Setup.hs} the same as \texttt{haskell-aura}?

\subsection{Arch Linux Specifics}\label{arch-linux-specifics}

\paragraph{ABS Package
Building/Installation}\label{abs-package-buildinginstallation}

\begin{itemize}
\itemsep1pt\parskip0pt\parsep0pt
\item
  There is no longer a \texttt{-M} option. All ABS package interaction
  is done through \texttt{-S}.
\item
  Installs prebuilt binaries available from Arch servers by default.
\item
  Build options:
\item
  If the user specifies \texttt{-\/-build}, the package will be built
  manually via the ABS.
\end{itemize}

\paragraph{AUR Package
Building/Installation}\label{aur-package-buildinginstallation}

\begin{itemize}
\itemsep1pt\parskip0pt\parsep0pt
\item
  Builds manually by default, as there is no prebuilt alternative for
  the AUR (by design).
\end{itemize}

\paragraph{PKGBUILD/Additional Build-file
Editing}\label{pkgbuildadditional-build-file-editing}

\begin{itemize}
\itemsep1pt\parskip0pt\parsep0pt
\item
  Support for \texttt{customizepkg} is dropped, as AUR 3.0 provides
  dependency information via its API.
\item
  Users can edit included \texttt{.install} files and the
  \textbf{behaviour} of PKGBUILDs with \texttt{-\/-edit}. This is done
  after dependency checks have been made via the data from the AUR API.
  Users are urged \emph{not} to edit dependencies at this point, as only
  \texttt{makepkg}, not Aura, will know about the changes.
\item
  If you do want to build a package with different dependencies,
  consider whether there is value in creating your own forked package
  for the AUR (named \texttt{foo-legacy}, etc.). Others may benefit from
  your effort.
\item
  If you are trying to fix a broken package, rather than circumventing
  the problem by building manually with \texttt{makepkg}, please contact
  the maintainer.
\end{itemize}

\paragraph{AUR Interaction}\label{aur-interaction}

\begin{itemize}
\itemsep1pt\parskip0pt\parsep0pt
\item
  AUR API calls are moved out of Aura and into a new Hackage package
  \texttt{aur} (exposing the \texttt{Linux.Arch.Aur.*} modules).
\item
  It provides conversions to and from JSON data and Haskell data.
\item
  This is preparation for future versions of Aura that allow use in
  other Linux distributions by swapping out sections of their back-end
  (with modules like \texttt{Linux.Debian.Repo} etc.)
\end{itemize}

\subsection{Coding Standards}\label{coding-standards}

\subsubsection{Record Syntax}\label{record-syntax}

When using record syntax for ADTs, function names should be suffixed
with ``Of'' to reflect their noun-like nature:

\begin{Shaded}
\begin{Highlighting}[]
\KeywordTok{data} \DataTypeTok{Package} \FunctionTok{=} \DataTypeTok{Package} \NormalTok{\{}\OtherTok{ nameOf    ::} \DataTypeTok{String}
                       \NormalTok{,}\OtherTok{ versionOf ::} \DataTypeTok{Version}
                       \NormalTok{,}\OtherTok{ depsOf    ::} \NormalTok{[}\DataTypeTok{Package}\NormalTok{] \}}
                       \KeywordTok{deriving} \NormalTok{(}\DataTypeTok{Eq}\NormalTok{, }\DataTypeTok{Show}\NormalTok{)}
\end{Highlighting}
\end{Shaded}

\end{document}
